\begin{minted}{python}
# author: Lukas Kluge, 2022-05-05

import math

def cubicTrajectory(psx1, psy1, nsx0, nsy0, nsx1, nsy1,
        nvNorm, mls, minT, maxPos, maxSpeed):
    #
    # Given information about two melt trajectories, this function
    # calculates a cubic move trajectory that starts at max velocity
    #
    #
    # inputs:
    #
    # psx1 = end x-coord of previous line segment [mm]
    # psy1 = end y-coord of previous line segment [mm]
    #
    # nsx0 = start x-coord of next line segment [mm]
    # nsy0 = start y-coord of next line segment [mm]
    #
    # nsx1 = end x-coord of next line segment [mm]
    # nsy1 = end y-coord of next line segment [mm]
    #
    # nvNorm = speed (norm of velocity) of next line
    #
    # mls = max. no. line segments of move trajectory (depends on GLAMS buffer size)
    # minT = min. duration of each line segment [s] (depends on scanner read speed)
    # 
    # maxPos = max. distance the trajectory can deviate from the center of field
    # maxSpeed = max. absolute velocity of the trajectory.
    # Should be at least three times higher than nvNorm (see t1x, t1y)
    #
    # outputs:
    # array of [x, y, speed] (units [mm, mm, mm/s]) ready for g-code
    #
    # if you don't like short var names use find/replace to change them

    t1 = endTime(psx1, psy1, nsx0, nsy0, nsx1, nsy1,
            nvNorm, maxSpeed)
    beta = cubicPolynomialCoeffs(psx1, psy1,
            nsx0, nsy0, nsx1, nsy1, nvNorm, t1)
    t = discreteTime(t1, mls, minT)
    points = len(t)
    sx = [0]*points
    sy = [0]*points
    for i in range(points):
        sx[i] = cubicPolynomial(t[i], beta[0][0], beta[0][1], beta[0][2],
                beta[0][3])
        sy[i] = cubicPolynomial(t[i], beta[1][0], beta[1][1], beta[1][2],
                beta[1][3])

    limitPosition(points, t, sx, sy, maxPos)
    sNorm = findDistances(points, sx, sy)
    vNorm = findSpeeds(points, t, sNorm)

    return [sx[1:], sy[1:], vNorm]

def endTime(psx1, psy1, nsx0, nsy0, nsx1, nsy1,
        nvNorm, maxSpeed):
    # The trajectory is assumed to start at t=0.
    # This functions calculates at what time it should end

    # find maxSpeed in each direction, taking into account
    # norm max speed and sign to get a positive t1
    maxSpeedx = 0.7071*math.copysign(maxSpeed, nsx0-psx1)
    maxSpeedy = 0.7071*math.copysign(maxSpeed, nsy0-psy1)
    
    nsNorm = math.sqrt((nsx1-nsx0)**2 + (nsy1-nsy0)**2)
    nvx = (nsx1-nsx0)/nsNorm*nvNorm # next vel in x-direction
    nvy = (nsy1-nsy0)/nsNorm*nvNorm # next vel in y-direction

    t1x = (3*nsx0-3*psx1)/(2*nvx+maxSpeedx)
    t1y = (3*nsy0-3*psy1)/(2*nvy+maxSpeedy)

    return t1x if (t1x>t1y) else t1y

def cubicPolynomialCoeffs(psx1, psy1, nsx0, nsy0, nsx1, nsy1,
        nvNorm, t1):
    # This calculates the coefficients of the polynomials describing the trajectory
    # TODO refactor into two functions: find boundary conditions and find coeffs

    nsNorm = math.sqrt((nsx1-nsx0)**2 + (nsy1-nsy0)**2)
    nvx = (nsx1-nsx0)/nsNorm*nvNorm # nrev vel in x-direction
    nvy = (nsy1-nsy0)/nsNorm*nvNorm # nrev vel in y-direction

    # x polynomial coefficients
    betax0 = psx1
    betax1 = -(3*psx1 - 3*nsx0 + 2*t1*nvx)/t1
    betax2 = (3*(psx1 - nsx0 + t1*nvx))/t1**2
    betax3 = -(psx1 - nsx0 + t1*nvx)/t1**3

    # y polynomial coefficients
    betay0 = psy1
    betay1 = -(3*psy1 - 3*nsy0 + 2*t1*nvy)/t1
    betay2 = (3*(psy1 - nsy0 + t1*nvy))/t1**2
    betay3 = -(psy1 - nsy0 + t1*nvy)/t1**3

    return [[betax0, betax1, betax2, betax3],
            [betay0, betay1, betay2, betay3]]

def discreteTime(t1, mls, minT):
    if (t1/minT <= mls):
        ls = math.floor(t1/minT)
        nomT = t1/ls # nominal T (actual dt might differ after limit correction)
        tDiscrete = [0]*(ls+1) # initialise
        for i in range(ls+1):
            tDiscrete[i] = nomT*i
    else:
        nomT = t1/mls # nominal T (actual dt might differ after limit correction)
        tDiscrete = [0]*(mls+1) # initialise
        for i in range(mls+1):
            tDiscrete[i] = nomT*i
    return tDiscrete

def cubicPolynomial(t, beta0, beta1, beta2, beta3):
    return beta0 + beta1*t + beta2*t**2 + beta3*t**3

def limitPosition(points, t, sx, sy, maxPos):
    for i in range(points):
        if sx[i] > maxPos:
            sx[i] = maxPos
        elif sx[i] < -maxPos:
            sx[i] = -maxPos
        if sy[i] > maxPos:
            sy[i] = maxPos
        elif sy[i] < -maxPos:
            sy[i] = -maxPos
    return [t, sx, sy]

def findDistances(points, sx, sy):
    sNorm = [0]*(points - 1)
    for i in range(points - 1):
        sNorm[i] = math.sqrt((sx[i+1]-sx[i])**2 + (sy[i+1]-sy[i])**2)
    return sNorm

def findSpeeds(points, t, sNorm):
    vNorm = [0]*(points - 1)
    for i in range(points - 1):
        T = t[i+1] - t[i]
        vNorm[i] = sNorm[i]/T
    return vNorm
\end{minted}