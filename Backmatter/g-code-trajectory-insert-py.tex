\begin{minted}{python}
# g-code-trajectory-insert

# usage in terminal: 
# $ py g-code-trajectory-insert.py "filename.g" 57.5 1000 600 63 0.00002 3
# order of arguments: maxPos maxSpeed cpAvgV mls minT style

import sys
import datetime

import quinticTrajectory as q
import cubicTrajectory as c

rawFileName = sys.argv[1]
maxPos = float(sys.argv[2])
maxSpeed = float(sys.argv[3])
cpAvgV = float(sys.argv[4]) # irrelevant for cubic
mls = int(sys.argv[5]) # GLAMS g-code buffer size
minT = float(sys.argv[6]) # less than read speed of scanner
style = int(sys.argv[7]) # 3 or 5 for cubic or quintic trajectories

def isMeltLine(line):
    coords = line.find("X")
    density = line.find("D")
    return (coords > -1) and (density > -1) and int(line[density+1]) > 0

def isMoveLine(line):
    coords = line.find("X")
    density = line.find("D")
    return (coords > -1) and ((density == -1) or int(line[density+1]) == 0)

def isSpeedChange(command):
    speed = -1
    for x in command:
        if x[0] == 'F':
            speed = float(x[1:])
    return speed

def findCoords(command):
    i = 0
    while i < len(command) and command[i][0] != "X":
        i += 1

    inputX = float(command[i][1:])
    inputY = float(command[i + 1][1:])
    return [inputX, inputY]

# create file with trajectories if not existing
fileWithTrajectories = open(rawFileName[:-2] + "-with-" + str(style) +
        "trajectories.g", 'x')

# read original file and fill new file accordingly
with open(rawFileName, 'r') as rawFile:
    firstEmptyLineFlag = True # checks where to add "trajectories added" note
    sa = [0, 0] # init most recent position [x [mm], y [mm]]
    sb = [-1, 0] # init second most recent position
    sc = [0, 0] # init third most recent position
    sd = [0, 0] # init fourth most recent position 
    pba = [0, True] # init path from sb to sa [speed, laser on?]
    pcb = [0, False] # init path from sc to sb
    pdc = [0, False] # init path from sd to sc
    cache = [sa, pba, sb, pcb, sc, pdc, sd] # array of previous movements
    currentSpeed = 0 # speed is set asynchroniously in the g-code, thus need var

    for line in rawFile:
        command = line.split() # divide into components

        if len(command) == 0:
            # add metadata about added trajectories,
            if firstEmptyLineFlag:
                fileWithTrajectories.write(";Trajectories added: " +
                        datetime.datetime.now().isoformat() + "\n")
                fileWithTrajectories.write(";Max position: " +
                        '%.5f' % maxPos + "\n")
                fileWithTrajectories.write(";Max speed: " +
                        '%.5f' % maxSpeed + "\n")
                fileWithTrajectories.write(";Min T (time for one movement): " +
                        '%.5f' % minT + "\n")
                fileWithTrajectories.write(";average speed of move trajectories: " +
                        '%.5f' % cpAvgV + "\n")
                fileWithTrajectories.write(";Max line segments: " +
                        '%.5f' % mls + "\n")
                fileWithTrajectories.write(";polynomial type (3rd or 5th order): " +
                        '%.5f' % style + "\n")

                firstEmptyLineFlag = False
            # copy (empty) line to output
            fileWithTrajectories.write(line)

        # copy lines that are not about movement directly to output
        elif command[0] != 'G1':
            fileWithTrajectories.write(line)

        # check if line moves the off laser and store info if it does
        elif isMoveLine(line):
            sd = sc
            sc = sb
            sb = sa
            sa = findCoords(command)

            pdc = pcb
            pcb = pba
            pba = [currentSpeed, False]

            if not pcb[1]:
                fileWithTrajectories.write(line) # TODO write previous line instead

        
        # check if line moves the on laser and print the appropriate
        # move trajectory to lead up to and including that line.
        elif isMeltLine(line):
            # make room for new position
            sd = sc
            sc = sb
            sb = sa
            sa = findCoords(command) # add newest position

            # make room for new path
            pdc = pcb
            pcb = pba
            pba = [currentSpeed, True] # add newest path

            if not pcb[1]: # check if previous line was move line 
                if style == 5: # quintic
                    moveTrajectory = q.quinticTrajectory(sd[0], sd[1], sc[0], sc[1],
                            sb[0], sb[1], sa[0], sa[1],
                            pdc[0], pba[0], cpAvgV, mls, minT, maxPos, maxSpeed)
                elif style == 3: # cubic
                    moveTrajectory = c.cubicTrajectory(sc[0], sc[1], sb[0], sb[1],
                            sa[0], sa[1], pba[0], mls, minT, maxPos, maxSpeed)

                # print move trajectory g-code and melt trajectory g-code
                for i in range(len(moveTrajectory[0])):
                    # speed
                    fileWithTrajectories.write(
                            "G1 F" + '%.6f' % moveTrajectory[2][i] + "\n")
                    # position (add D at the end if the laser should be on)
                    fileWithTrajectories.write(
                            "G1 X" + '%.6f' % moveTrajectory[0][i] + " Y" +
                                    '%.6f' % moveTrajectory[1][i] + "\n")
                
                fileWithTrajectories.write("G1 F" + str(pba[0]) + "\n")
                fileWithTrajectories.write(line)

        # other G-commands should there be any,
        # are just copied to output file
        else: 
            if isSpeedChange(command) > -1: 
                currentSpeed = isSpeedChange(command)
            fileWithTrajectories.write(line)

fileWithTrajectories.close()

\end{minted}