\begin{minted}{python}
# author: Lukas Kluge, 2022-05-05

import math

def quinticTrajectory(psx0, psy0, psx1, psy1, nsx0, nsy0, nsx1, nsy1,
        pvNorm, nvNorm, cpAvgV, mls, minT, maxPos, maxSpeed):
    #
    # Given information about two melt trajectories, this function
    # calculates a quintic move trajectory that connects them smoothly
    #
    # inputs:
    # psx0 = start x-coord of previous line segment [mm]
    # psy0 = start y-coord of previous line segment [mm]
    #
    # psx1 = end x-coord of previous line segment [mm]
    # psy1 = end y-coord of previous line segment [mm]
    #
    # nsx0 = start x-coord of next line segment [mm]
    # nsy0 = start y-coord of next line segment [mm]
    #
    # nsx1 = end x-coord of next line segment [mm]
    # nsy1 = end y-coord of next line segment [mm]
    #
    # pvNorm = speed (norm of velocity) of previous line
    # nvNorm = speed of next line
    #
    # cpAvgV = desired absolute average velocity of move trajectory [mm/s]
    #
    # mls = max. no. line segments of move trajectory (depends on GLAMS buffer size)
    # minT = min. duration of each line segment [s] (depends on scanner read speed)
    # 
    # maxPos = max. distance the trajectory can deviate from the center of field
    # maxSpeed = max. absolute velocity of the trajectory
    #
    # outputs:
    # array of [x, y, speed] (units [mm, mm, mm/s]) ready for g-code
    #
    # if you don't like short var names use find/replace to change them

    t1 = endTime(psx1, psy1, nsx0, nsy0, cpAvgV)
    beta = quinticPolynomialCoeffs(psx0, psy0, psx1, psy1,
            nsx0, nsy0, nsx1, nsy1, pvNorm, nvNorm, t1)
    t = discreteTime(t1, mls, minT)
    points = len(t)
    sx = [0]*points
    sy = [0]*points
    for i in range(points):
        sx[i] = quinticPolynomial(t[i], beta[0][0], beta[0][1], beta[0][2],
                beta[0][3], beta[0][4], beta[0][5])
        sy[i] = quinticPolynomial(t[i], beta[1][0], beta[1][1], beta[1][2],
                beta[1][3], beta[1][4], beta[1][5])

    limitPosition(points, t, sx, sy, maxPos)
    sNorm = findDistances(points, sx, sy)
    vNorm = findSpeeds(points, t, sNorm)
    limitSpeeds(points, vNorm, t, sNorm, maxSpeed)

    return [sx[1:], sy[1:], vNorm]

def endTime(psx1, psy1, nsx0, nsy0, cpAvgV):
    # The trajectory is assumed to start at t=0.
    # This functions calculates at what time it should end
    cpsNorm = math.sqrt((nsx0-psx1)**2 + (nsy0-psy1)**2)
    return cpsNorm/cpAvgV

def quinticPolynomialCoeffs(psx0, psy0, psx1, psy1, nsx0, nsy0, nsx1, nsy1,
        pvNorm, nvNorm, t1):
    # This calculates the coefficients of the polynomials describing the trajectory
    # TODO refactor into two functions: find boundary conditions and find coeffs

    psNorm = math.sqrt((psx1-psx0)**2 + (psy1-psy0)**2) 
    pvx = (psx1-psx0)/psNorm*pvNorm # prev vel in x-direction
    pvy = (psy1-psy0)/psNorm*pvNorm # prev vel in y-direction

    nsNorm = math.sqrt((nsx1-nsx0)**2 + (nsy1-nsy0)**2)
    nvx = (nsx1-nsx0)/nsNorm*nvNorm # nrev vel in x-direction
    nvy = (nsy1-nsy0)/nsNorm*nvNorm # nrev vel in y-direction

    # x polynomial coefficients
    betax0 = psx1
    betax1 = pvx
    betax2 = 0
    betax3 = -(10*psx1 - 10*nsx0 + t1*(6*pvx + 4*nvx))/(t1**3)
    betax4 = (15*psx1 - 15*nsx0 + t1*(8*pvx + 7*nvx))/(t1**4)
    betax5 = -(6*psx1 - 6*nsx0 + t1*(3*pvx + 3*nvx))/(t1**5)

    # y polynomial coefficients
    betay0 = psy1
    betay1 = pvy
    betay2 = 0
    betay3 = -(10*psy1 - 10*nsy0 + t1*(6*pvy + 4*nvy))/(t1**3)
    betay4 = (15*psy1 - 15*nsy0 + t1*(8*pvy + 7*nvy))/(t1**4)
    betay5 = -(6*psy1 - 6*nsy0 + t1*(3*pvy + 3*nvy))/(t1**5)

    return [[betax0, betax1, betax2, betax3, betax4, betax5],
            [betay0, betay1, betay2, betay3, betay4, betay5]]

def discreteTime(t1, mls, minT):
    if (t1/minT <= mls):
        ls = math.floor(t1/minT)
        nomT = t1/ls # nominal T (actual dt might differ after limit correction)
        tDiscrete = [0]*(ls+1) # initialise
        for i in range(ls+1):
            tDiscrete[i] = nomT*i
    else:
        nomT = t1/mls # nominal T (actual dt might differ after limit correction)
        tDiscrete = [0]*(mls+1) # initialise
        for i in range(mls+1):
            tDiscrete[i] = nomT*i
    return tDiscrete

def quinticPolynomial(t, beta0, beta1, beta2, beta3, beta4, beta5):
    return beta0 + beta1*t + beta2*t**2 + beta3*t**3 + beta4*t**4 + beta5*t**5

def limitPosition(points, t, sx, sy, maxPos):
    for i in range(points):
        if sx[i] > maxPos:
            sx[i] = maxPos
        elif sx[i] < -maxPos:
            sx[i] = -maxPos
        if sy[i] > maxPos:
            sy[i] = maxPos
        elif sy[i] < -maxPos:
            sy[i] = -maxPos
    return [t, sx, sy]

def findDistances(points, sx, sy):
    sNorm = [0]*(points - 1)
    for i in range(points - 1):
        sNorm[i] = math.sqrt((sx[i+1]-sx[i])**2 + (sy[i+1]-sy[i])**2)
    return sNorm

def findSpeeds(points, t, sNorm):
    vNorm = [0]*(points - 1)
    for i in range(points - 1):
        T = t[i+1] - t[i]
        vNorm[i] = sNorm[i]/T
    return vNorm

def limitSpeeds(points, vNorm, t, sNorm, maxSpeed):
    Tnom = t[1] # nominal time interval (before speed limit)
    for i in range(points-1):
        if vNorm[i] > maxSpeed:
            vNorm[i] = maxSpeed
            t[i+1] = t[i] + sNorm[i]/maxSpeed
            for j in range((i+1),points-1):
                t[j] = t[j] + (sNorm[j]/maxSpeed - Tnom)
    return vNorm

\end{minted}